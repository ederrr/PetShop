\chapter{Páginas}
Nesse capítulo descreveremos todas as páginas feitas no trabalho, lembrando
que elas não têm funcionalidade, só fazem parte do \emph{mockup}. Nelas contém
\emph{links}, mas são meramente demonstrativos e todo o conteúdo colocado em
formulários ou tela de \emph{login} tem o seu conteúdo enviado, mas não
armazenado.

Como foi utilizado o estilo \emph{Single-Page Application} todas as páginas têm
algumas informações em comum. Dentre elas, temos:
\begin{description}[style=nextline]
	\item [Cabeçalho]	O cabeçalho é composto por um logo do petshop, área do
						usuário, onde ele pode acessar a conta dele, ver os
						agendamentos e o carrinho dele. Além disso, o cabeçalho
						também tem um menu, que será descrito no próximo tópico.
	\item [Menu]		O menu é uma forma mais intuitiva de se navegar pelo site,
						ele é segmentado em três partes essenciais, o Início,
						Serviços e Animais. O Início é somente um \emph{link} para
						a página inicial do site, ou seja, de onde o usuário estiver
						ele pode ir para a página inicial clicando em Início.
						Serviços, é um acesso direto aos serviços que o usuário pode
						agendar. Por último, a sessão de Animais é composta por
						vários tipos de animais, como cachorro, gato, cavalo, etc.
						Se o usuário clicar em um desses animais, ele será
						redirecionado para a área de produtos para esse tipo de animal.
	\item [Rodapé]		No rodapé, há o nome completo de todos os integrantes do grupo,
						podendo ter futuramente um redirecionamento para a página do
						GitHub de cada um.
\end{description}

\section{Administrador}
Essa página representa como seria a página de um administrador, onde ele tem
as informações pessoais dele, como se fosse um usuário normal, e seus animais.
Além das informações pessoais do administrador, há sessões de produtos,
serviços e usuários, nos quais ele pode


admin.html, admin.css e admn.js
\section{Agendar}
Contém agendar.html, agendar.js e agendar.css
\section{Animal}
Contém animal.css e animal.html
\section{Cadastro Animal}
Contém cadastro-animal.css e cadastro-animal.html
\section{Cadastro Pessoal}
Contém cadastro-pessoal.css e cadastro-pessoal.html
\section{Cadastro Produto}
Contém cadastro-produto.css e cadastro-produto.html
\section{Cadastro Serviço}
Contém cadastro-servico.css e cadastro-servico.html
\section{Carrinho}
Contém carrinho.css e carrinho.html
\section{Espécie}
Contém especie.css e especie.html
\section{Index}
Contém index.css, index.html e index.js
\section{Produto}
Contém produto.css, produto.html e produto.js
\section{Serviço}
Contém servico.css, servico.html e servico.js
\section{Usuário}
Contém usuario.css, usuario.html e usuario.js