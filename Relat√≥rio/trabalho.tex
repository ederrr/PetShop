%%%%%%%%%%%%%%%%%%%%%%%%%%%%%%%%%%%%%%%%%%%%%%%%%%%%%%%%%%%
%			Projeto 1 - Mockup Petshop
%
%	Introdução ao Desenvolvimento Web			SCC0219
%	Prof. Dilvan de Abreu Moreira
%
%	Eder Rosati Ribeiro							8122585
%	Leonardo de Almeida Lima Zanguetin			8531866
%	Victor Luiz da Silva Mariano Pereira		8602444
%
%%%%%%%%%%%%%%%%%%%%%%%%%%%%%%%%%%%%%%%%%%%%%%%%%%%%%%%%%%%

% ---
% Definição do documento
% ---
\documentclass[
	% -- opções da classe memoir --
	12pt,				% tamanho da fonte
	openright,			% capítulos começam em pág ímpar
						% (insere página vazia caso preciso)
	%openany,			% capítulos começam em qualquer página
	twoside,			% para impressão em recto e verso. Oposto a oneside
	%oneside,			% para impressão somente em recto
	a4paper,			% tamanho do papel.
]{abntex2}
% ---

% ---
% Pacotes básicos
% ---
\usepackage{fontspec}
\usepackage{polyglossia}
	\setdefaultlanguage{portuges}
\usepackage{lmodern}			% Usa a fonte Latin Modern
\usepackage{indentfirst}		% Indenta o primeiro parágrafo de cada seção.
\usepackage{color}				% Controle das cores
\usepackage{graphicx}			% Inclusão de gráficos
	\graphicspath{{Figuras/}}
\usepackage{microtype}			% para melhorias de justificação
\usepackage{amssymb}
% ---

% ---
% Pacotes de citações
% ---
\usepackage[brazilian,hyperpageref]{backref}	% Paginas com as citações na
												% bibliografia
\usepackage[alf]{abntex2cite}	% Citações padrão ABNT
\usepackage{caption}			% Pacote para resolver o problema do
								% "listings", "minted" e códigos muito grandes
\usepackage{listings}			% Pacote para listagens de códigos
\usepackage{minted}				% Inserção de códigos fonte
% ---

% ---
% Pacotes para tabelas
% ---
\usepackage{longtable}
\usepackage{booktabs}
\usepackage{array}
% ---

% ---
% Pacotes adicionais
% ---
%\usepackage{showframe}			% Pacote para mostrar as caixas de texto
% ---

% ---
% Informações de dados para CAPA e FOLHA DE ROSTO
% ---
\titulo{Projeto 1 \\ \emph{Mockup} PetShop}
\autor{
		Eder Rosati Ribeiro						---	8122585 \and\\
		Leonardo de Almeida Lima Zanguetin		---	8531866	\and\\
		Victor Luiz da Silva Mariano Pereira	---	8602444 \\
}
\local{Brasil}
\data{2018}
\instituicao{%
  Universidade de São Paulo -- USP
  \par
  Instituto de Ciências Matemáticas e de Computação -- ICMC
  \par
  Introdução ao Desenvolvimento Web -- SCC0219}
\tipotrabalho{Trabalho Acadêmico}
% ---

% ---
% Configurações de aparência do PDF final

% alterando o aspecto da cor azul
\definecolor{blue}{RGB}{41,5,195}

% informações do PDF
\makeatletter
\hypersetup{
	%pagebackref=true,
	pdftitle={\@title},
	pdfauthor={\@author},
	pdfsubject={Mockup PetShop},
	pdfcreator={LuaLaTeX with abnTeX2},
	pdfkeywords={USP }{ICMC }{Introdução ao Desenvolvimento Web }{Mockup }{PetShop},
	colorlinks=true,			% false: boxed links; true: colored links
	linkcolor=blue,				% color of internal links
	citecolor=blue,				% color of links to bibliography
	filecolor=magenta,			% color of file links
	urlcolor=blue,
	bookmarksdepth=4
}
\makeatother
% ---

% ---
% Espaçamentos entre linhas e parágrafos
% ---

% O tamanho do parágrafo é dado por:
\setlength{\parindent}{1.3cm}

% Controle do espaçamento entre um parágrafo e outro:
\setlength{\parskip}{0.2cm}  % tente também \onelineskip

% ---
% compila o indice
% ---
\makeindex
% ---

% ----
% Início do documento
% ----
\begin{document}

% Retira espaço extra obsoleto entre as frases.
\frenchspacing

% ----------------------------------------------------------
% ELEMENTOS PRÉ-TEXTUAIS
% ----------------------------------------------------------
\pretextual

% ---
% Capa
% ---
\imprimircapa
% ---

% ---
% Folha de rosto
% (o * indica que haverá a ficha bibliográfica)
% ---
\imprimirfolhaderosto*
% ---

% ---
% RESUMOS
% ---
% resumo em português
%\setlength{\absparsep}{18pt} % ajusta o espaçamento dos parágrafos do resumo
%\begin{resumo}
%\end{resumo}

% ---
% inserir lista de ilustrações
% ---
%\pdfbookmark[0]{\listfigurename}{lof}
%\listoffigures*
%\clearpage
% ---

% ---
% inserir lista de tabelas
% ---
%\pdfbookmark[0]{\listtablename}{lot}
%\listoftables*
%\clearpage
% ---

% ---
% inserir lista de códigos fonte
% ---
%\renewcommand\listoflistingscaption{Lista de Códigos Fonte}
%\listoflistings
%\clearpage
% ---

% ---
% inserir o sumario
% ---
%\pdfbookmark[0]{\contentsname}{toc}
%\tableofcontents*
%\clearpage
% ---

% ----------------------------------------------------------
% ELEMENTOS TEXTUAIS
% ----------------------------------------------------------
\textual

% ----------------------------------------------------------
% Corpo do trabalho
% ----------------------------------------------------------
\chapter{Introdução}

Nesta fase inicial da aplicação PetShop, nos foi solicitado que com as
linguagens \textsc{html5} e \textsc{css3} fosse desenvolvido o \emph{mockup}
funcional deste, ou seja, o “esqueleto” do que será ao final nosso \emph{site},
seu visual; a estrutura do que cada pagina conterá; as “funcionalidades”
possíveis através do \emph{site} as informações que ele conterá em suas páginas
internas.

Esta fase da criação desse protótipo é muito importante para o desenvolvimento,
este nada mais é do que um exemplo do que se tornará o projeto ao seu fim,
porém com seu uso podemos observar este modelo prévio e avaliá-lo, obtermos
\emph{feedback} de clientes e futuros usuários e encontrar erros ou incongruências
que se fazem de mais fácil correção neste momento do que no futuro onde o
projeto estará em uma fase mais complexa e que qualquer falha, mesmo que
pequena, pode atrasar ou torná-lo muito trabalhoso para ser concertado.

Então por fim podemos ver que com esta parte do projeto bem executada, as
próximas se tornam muito mais simples do que seriam se a criação do \emph{mockup},
outro elemento que também desenvolvemos, foi o esboço prévio da página
principal desenhado em papel, este é um modo de fazer com que a criação da
pévia seja ainda mais simples, pois já se tem a parte gráfica criada pra se
basear, depois disso sé se faz necessária a tradução daquilo para a linguagem
de marcação.
\chapter{Páginas}
Nesse capítulo descreveremos todas as páginas feitas no trabalho, lembrando
que elas não têm funcionalidade, só fazem parte do \emph{mockup}. Nelas contém
\emph{links}, mas são meramente demonstrativos e todo o conteúdo colocado em
formulários ou tela de \emph{login} tem o seu conteúdo enviado, mas não
armazenado.

Como foi utilizado o estilo \emph{Single-Page Application} todas as páginas têm
algumas informações em comum. Dentre elas, temos:
\begin{description}[style=nextline]
	\item [Cabeçalho]	O cabeçalho é composto por um logo do petshop, área do
						usuário, onde ele pode acessar a conta dele, ver os
						agendamentos e o carrinho dele. Além disso, o cabeçalho
						também tem um menu, que será descrito no próximo tópico.
	\item [Menu]		O menu é uma forma mais intuitiva de se navegar pelo site,
						ele é segmentado em três partes essenciais, o Início,
						Serviços e Animais. O Início é somente um \emph{link} para
						a página inicial do site, ou seja, de onde o usuário estiver
						ele pode ir para a página inicial clicando em Início.
						Serviços, é um acesso direto aos serviços que o usuário pode
						agendar. Por último, a sessão de Animais é composta por
						vários tipos de animais, como cachorro, gato, cavalo, etc.
						Se o usuário clicar em um desses animais, ele será
						redirecionado para a área de produtos para esse tipo de animal.
	\item [Rodapé]		No rodapé, há o nome completo de todos os integrantes do grupo,
						podendo ter futuramente um redirecionamento ao clicar no nome
						de cada um para a página do	GitHub do mesmo.
\end{description}

\section{Administrador}
Essa página representa como seria a página de um administrador, onde ele tem
as informações pessoais dele, como se fosse um usuário normal, e seus animais.

Além das informações pessoais do administrador, há sessões de produtos,
serviços e usuários, nos quais ele pode fazer busca por algum produto (por nome,
código ou categoria), serviço (por nome, id, categoria) ou usuários (por nome ou
CPF).

Para cada produto, serviço ou usuário, já criado, é possível editar as informações
ou excluir. Também é possível que o administrador adicione um novo de qualquer uma
dessas categorias.

Por fim, na página de administrador, também há um registro de compras realizadas por
ele, contendo uma tabela com produto, data, valor de cada produto, e o valor total da
compra. E um registro de serviços agendados por ele, que contém o tipo de serviço, o
animal para que foi registrado o serviço, a data do serviço, valor de cada serviço e o
valor total de serviços.

Os que compõem a página de administrador são: \texttt{admin.html}, \texttt{admin.css} e \texttt{admin.js}

\section{Agendar}
Contém agendar.html, agendar.js e agendar.css
\section{Animal}
Contém animal.css e animal.html
\section{Cadastro Animal}
Contém cadastro-animal.css e cadastro-animal.html
\section{Cadastro Pessoal}
Contém cadastro-pessoal.css e cadastro-pessoal.html
\section{Cadastro Produto}
Contém cadastro-produto.css e cadastro-produto.html
\section{Cadastro Serviço}
Contém cadastro-servico.css e cadastro-servico.html
\section{Carrinho}
Contém carrinho.css e carrinho.html
\section{Espécie}
Contém especie.css e especie.html
\section{Index}
Contém index.css, index.html e index.js
\section{Produto}
Contém produto.css, produto.html e produto.js
\section{Serviço}
Contém servico.css, servico.html e servico.js
\section{Usuário}
Contém usuario.css, usuario.html e usuario.js
\chapter{Considerações finais}
% ----------------------------------------------------------

% ----------------------------------------------------------
% ELEMENTOS PÓS-TEXTUAIS
% ----------------------------------------------------------
\postextual
% ----------------------------------------------------------

% ----------------------------------------------------------
% Referências bibliográficas
% ----------------------------------------------------------
%\bibliography{trabalho}

\end{document}
