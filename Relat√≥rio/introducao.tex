\chapter{Introdução}

Nesta fase inicial da aplicação PetShop, nos foi solicitado que com as
linguagens \textsc{html5} e \textsc{css3} fosse desenvolvido o \emph{mockup}
funcional deste, ou seja, o “esqueleto” do que será ao final nosso \emph{site},
seu visual; a estrutura do que cada pagina conterá; as “funcionalidades”
possíveis através do \emph{site} as informações que ele conterá em suas páginas
internas.

Esta fase da criação desse protótipo é muito importante para o desenvolvimento,
este nada mais é do que um exemplo do que se tornará o projeto ao seu fim,
porém com seu uso podemos observar este modelo prévio e avaliá-lo, obtermos
\emph{feedback} de clientes e futuros usuários e encontrar erros ou incongruências
que se fazem de mais fácil correção neste momento do que no futuro onde o
projeto estará em uma fase mais complexa e que qualquer falha, mesmo que
pequena, pode atrasar ou torná-lo muito trabalhoso para ser concertado.

Então por fim podemos ver que com esta parte do projeto bem executada, as
próximas se tornam muito mais simples do que seriam se a criação do \emph{mockup},
outrp elemento que também desenvolvemos, foi o esboço prévio da página
principal desenhado em papel, este é um modo de fazer com que a criação da
pévia seja ainda mais simples, pois já se tem a parte gráfica criada pra se
basear, depois disso sé se faz necessária a tradução daquilo para a linguagem
de marcação.