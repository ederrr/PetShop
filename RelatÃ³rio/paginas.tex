\chapter{Páginas}
Nesse capítulo descreveremos todas as páginas feitas no trabalho, lembrando
que elas não têm funcionalidade, só fazem parte do \emph{mockup}. Nelas contém
\emph{links}, mas são meramente demonstrativos e todo o conteúdo colocado em
formulários ou tela de \emph{login} tem o seu conteúdo enviado, mas não
armazenado.

Como foi utilizado o estilo \emph{Single-Page Application} todas as páginas têm
algumas informações em comum. Dentre elas, temos:
\begin{description}[style=nextline]
	\item [Cabeçalho]	O cabeçalho é composto por um logo do petshop, área do
						usuário, onde ele pode acessar a conta dele, ver os
						agendamentos e o carrinho dele. Além disso, o cabeçalho
						também tem um menu, que será descrito no próximo tópico.
	\item [Menu]		O menu é uma forma mais intuitiva de se navegar pelo site,
						ele é segmentado em três partes essenciais, o Início,
						Serviços e Animais. O Início é somente um \emph{link} para
						a página inicial do site, ou seja, de onde o usuário estiver
						ele pode ir para a página inicial clicando em Início.
						Serviços, é um acesso direto aos serviços que o usuário pode
						agendar. Por último, a sessão de Animais é composta por
						vários tipos de animais, como cachorro, gato, cavalo, etc.
						Se o usuário clicar em um desses animais, ele será
						redirecionado para a área de produtos para esse tipo de animal.
	\item [Rodapé]		No rodapé, há o nome completo de todos os integrantes do grupo,
						podendo ter futuramente um redirecionamento ao clicar no nome
						de cada um para a página do	GitHub do mesmo.
\end{description}

\section{Administrador}
Essa página representa como seria a página de um administrador, onde ele tem
as informações pessoais dele, como se fosse um usuário normal, e seus animais.

Além das informações pessoais do administrador, há sessões de produtos,
serviços e usuários, nos quais ele pode fazer busca por algum produto (por nome,
código ou categoria), serviço (por nome, id, categoria) ou usuários (por nome ou
\textsc{cpf}).

Para cada produto, serviço ou usuário, já criado, é possível editar as informações
ou excluir. Também é possível que o administrador adicione um novo de qualquer uma
dessas categorias.

Por fim, na página de administrador, também há um registro de compras realizadas por
ele, contendo uma tabela com produto, data, valor de cada produto, e o valor total da
compra. E um registro de serviços agendados por ele, que contém o tipo de serviço, o
animal para que foi registrado o serviço, a data do serviço, valor de cada serviço e o
valor total de serviços.

Os arquivos que compõem a página de administrador são: \texttt{admin.html},
\texttt{admin.css} e \texttt{admin.js}.

\section{Agendar}
Na página de agendamento de serviço, tem o tipo de serviço que o usuário escolheu,
podendo ser banho, tosa, consulta veterinária e hospedagem. Seguido de uma breve descrição
do serviço que poderá ser agendado.

Na sessão logo a baixo da descrição há uma lista de animais cadastrados na sua conta
que será para ser feita a seleção para qual animal será o serviço. E ao lado um campo
para a escolha do dia e horário do agendamento, valor do serviço e um botão para confirmar
o agendamento.

Os arquivos que compõem a página para agendamento são: \texttt{agendar.html},\\
\texttt{agendar.js} e \texttt{agendar.css}.

\section{Animal}
A página de animal, contém as informações de um animal registrado por um usuário, tais como
foto, nome, idade, espécie, raça, porte, sexo, pelagem, \textsc{rga} (Registro Geral Animal)
e peso.

Numa sessão logo abaixo tem o histórico de agendamento de serviços feitos para esse animal,
com o tipo de serviço, data do serviço, o valor de cada serviço e o valor total gasto com
esse animal.

Os arquivos que compõem a página animal são: \texttt{animal.css} e \texttt{animal.html}

\section{Cadastros}
As páginas de cadastros são separadas em quatro tipos, os quais são: cadastro animal,
cadastro pessoal, cadastro de produto e cadastro de serviço.

Essas páginas são compostas por formulários, onde o usuário ou o administrador pode preencher
para poder cadastrar o que quer e um botão para enviar o conteúdo do formulário. Obviamente
que cadastro de produtos e cadastro de serviços são páginas exclusivas de administrador.

\subsection{Cadastro Animal}
A página de cadastro animal, como fora dito anteriormente, consiste de um formulário,
onde o usuário pode cadastrar o seu animal de estimação, preenchendo as informações e
clicando no botão para o cadastro.

Nesse formulário há a possibilidade do usuário preencher com nome, foto do animal, sexo,
tipo do animal, raça, pelagem ou cor, \textsc{rga}, porte, data de nascimento e peso.

Os arquivos que compõem a página cadastro animal são: \texttt{cadastro-animal.css} e
\texttt{cadçastro-animal.html}.

\subsection{Cadastro Pessoal}
Para o cadastro pessoal, temos uma página com um formulário, como a de todos os tipos
de cadastro no site, mas tendo apenas informações pessoais.

Para melhor estruturação semântica, o formulário é dividido em duas partes, uma para
informações própria do usuário, como: nome, e-mail, senha de acesso, \textsc{cpf},
data de nascimento, sexo, foto e telefone. E outra parte com informações do endereço
do mesmo, como: endereço, número, \textsc{cep}, complemento, bairro, cidade, estado
e um campo para a descrição de um ponto de referência, para ficar melhor para quem for
prestar o serviço de entrega ou “leva-e-trás”.

Os arquivos que compõem a página cadastro pessoal são: \texttt{cadastro-pessoal.css} e
\texttt{cadastro-pessoal.html}.

\subsection{Cadastro de Produto}
No cadastro do produto, como foi explicado anteriormente, é uma página que só o administrador
tem acesso. Nela o administrador pode preencher um formulário para adicionar um produto no
banco de dados.

O formulário é composto pelas seguintes informações: nome, id, preço, foto, categoria, marca,
peso líquido, tipo de embalagem, dimensões da embalagem, quantidade e uma área para descrição
do produto.

Os arquivos que compõem a página cadastro de produto são: \texttt{cadastro-produto.css} e
\texttt{cadastro-produto.html}.

\subsection{Cadastro de Serviço}
Para o cadastro de serviço, o administrador terá um formulário para preencher com as seguintes
informações: nome, id, preço, categoria, animais possíveis (essa área é para ter controle de um
usuário tentar agendar um serviço de tosa em uma iguana) e um local para uma breve descrição
do serviço.

Lembrando que essa página terá acesso reestrito, somente o administrador que poderá acessá-la.

Os arquivos que compõem a página cadastro de serviço são: \texttt{cadastro-servico.css} e
\texttt{cadastro-servico.html}.

\section{Carrinho}
A página carrinho é composta por todos os produtos que o usuário clicou em comprar. Nela temos
uma visão enxuta do produto, com a foto do produto, o valor dele e quantidade desejada.

A quantidade pode ser alterada pelo usuário, assim como o próprio produto, podendo excluí-lo se
desejar. Ao finalizar a compra, o usuário tem que clicar em fechar carrinho para concluir a compra.

Obs.: Ainda não foi implementada a parte final para a confirmação da compra com o pagamento.

Os arquivos que compõem a página carrinho são: \texttt{carrinho.css} e \texttt{carrinho.html}.

\section{Espécie}
A página de espécie é composta por todos os produtos para aquele tipo de animal. Cada produto é
composto por uma foto do produto, o nome do produto e o preço dele.

Se clicarmos em algum produto, ele será redirecionado à página do produto que clicamos. Vale lembrar
que no site que fizemos temos imagens, produtos, valores tudo meramente ilustrativo, pois o
objetivo é mostrar como o site funcionaria e o \emph{layout} dele.

Os arquivos que compõem a página espécie são: \texttt{especie.css} e \texttt{especie.html}.

\section{Index}
O index, nada mais é do que a página inicial do nosso site. Nela temos os itens normais de todas as
páginas, como dissemos no começo desse capítulo, como cabeçalho, menu, área do usuário.

Além dessa área comum, também temos logo a baixo, uma sessão de slides que mostra os destaques como
serviços ou produtos em promoção, ou alguma ação que o petshop fez e queira divugar na página inicial.
Ainda não está funcionando automaticamente a parte de slides, mas basta clicar nos quadradinhos que a
imagem muda.

Em seguida, temos os produtos em destaque de cada tipo de animal em forma de grade, contendo a foto,
o nome do produto e o valor dele. Para uma próxima etapa, podemos fazer essa parte de produtos em
destaque de acordo com os animais que o usuário tem e os produtos que ele mais consome.

Os arquivos que compõem a página index são: \texttt{index.css}, \texttt{index.html} e
\texttt{index.js}.

\section{Produto}

Os arquivos que compõem a página produto são: \texttt{produto.css}, \texttt{produto.html} e
\texttt{produto.js}.

\section{Serviço}

Os arquivos que compõem a página serviço são: \texttt{servico.css}, \texttt{servico.html} e
\texttt{servico.js}.

\section{Usuário}

Os arquivos que compõem a página usuário são: \texttt{usuario.css}, \texttt{usuario.html} e
\texttt{usuario.js}.
